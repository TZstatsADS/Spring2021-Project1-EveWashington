% Options for packages loaded elsewhere
\PassOptionsToPackage{unicode}{hyperref}
\PassOptionsToPackage{hyphens}{url}
%
\documentclass[
]{article}
\usepackage{lmodern}
\usepackage{amssymb,amsmath}
\usepackage{ifxetex,ifluatex}
\ifnum 0\ifxetex 1\fi\ifluatex 1\fi=0 % if pdftex
  \usepackage[T1]{fontenc}
  \usepackage[utf8]{inputenc}
  \usepackage{textcomp} % provide euro and other symbols
\else % if luatex or xetex
  \usepackage{unicode-math}
  \defaultfontfeatures{Scale=MatchLowercase}
  \defaultfontfeatures[\rmfamily]{Ligatures=TeX,Scale=1}
\fi
% Use upquote if available, for straight quotes in verbatim environments
\IfFileExists{upquote.sty}{\usepackage{upquote}}{}
\IfFileExists{microtype.sty}{% use microtype if available
  \usepackage[]{microtype}
  \UseMicrotypeSet[protrusion]{basicmath} % disable protrusion for tt fonts
}{}
\makeatletter
\@ifundefined{KOMAClassName}{% if non-KOMA class
  \IfFileExists{parskip.sty}{%
    \usepackage{parskip}
  }{% else
    \setlength{\parindent}{0pt}
    \setlength{\parskip}{6pt plus 2pt minus 1pt}}
}{% if KOMA class
  \KOMAoptions{parskip=half}}
\makeatother
\usepackage{xcolor}
\IfFileExists{xurl.sty}{\usepackage{xurl}}{} % add URL line breaks if available
\IfFileExists{bookmark.sty}{\usepackage{bookmark}}{\usepackage{hyperref}}
\hypersetup{
  pdftitle={Party Perception's Role in Elections},
  hidelinks,
  pdfcreator={LaTeX via pandoc}}
\urlstyle{same} % disable monospaced font for URLs
\usepackage[margin=1in]{geometry}
\usepackage{color}
\usepackage{fancyvrb}
\newcommand{\VerbBar}{|}
\newcommand{\VERB}{\Verb[commandchars=\\\{\}]}
\DefineVerbatimEnvironment{Highlighting}{Verbatim}{commandchars=\\\{\}}
% Add ',fontsize=\small' for more characters per line
\usepackage{framed}
\definecolor{shadecolor}{RGB}{248,248,248}
\newenvironment{Shaded}{\begin{snugshade}}{\end{snugshade}}
\newcommand{\AlertTok}[1]{\textcolor[rgb]{0.94,0.16,0.16}{#1}}
\newcommand{\AnnotationTok}[1]{\textcolor[rgb]{0.56,0.35,0.01}{\textbf{\textit{#1}}}}
\newcommand{\AttributeTok}[1]{\textcolor[rgb]{0.77,0.63,0.00}{#1}}
\newcommand{\BaseNTok}[1]{\textcolor[rgb]{0.00,0.00,0.81}{#1}}
\newcommand{\BuiltInTok}[1]{#1}
\newcommand{\CharTok}[1]{\textcolor[rgb]{0.31,0.60,0.02}{#1}}
\newcommand{\CommentTok}[1]{\textcolor[rgb]{0.56,0.35,0.01}{\textit{#1}}}
\newcommand{\CommentVarTok}[1]{\textcolor[rgb]{0.56,0.35,0.01}{\textbf{\textit{#1}}}}
\newcommand{\ConstantTok}[1]{\textcolor[rgb]{0.00,0.00,0.00}{#1}}
\newcommand{\ControlFlowTok}[1]{\textcolor[rgb]{0.13,0.29,0.53}{\textbf{#1}}}
\newcommand{\DataTypeTok}[1]{\textcolor[rgb]{0.13,0.29,0.53}{#1}}
\newcommand{\DecValTok}[1]{\textcolor[rgb]{0.00,0.00,0.81}{#1}}
\newcommand{\DocumentationTok}[1]{\textcolor[rgb]{0.56,0.35,0.01}{\textbf{\textit{#1}}}}
\newcommand{\ErrorTok}[1]{\textcolor[rgb]{0.64,0.00,0.00}{\textbf{#1}}}
\newcommand{\ExtensionTok}[1]{#1}
\newcommand{\FloatTok}[1]{\textcolor[rgb]{0.00,0.00,0.81}{#1}}
\newcommand{\FunctionTok}[1]{\textcolor[rgb]{0.00,0.00,0.00}{#1}}
\newcommand{\ImportTok}[1]{#1}
\newcommand{\InformationTok}[1]{\textcolor[rgb]{0.56,0.35,0.01}{\textbf{\textit{#1}}}}
\newcommand{\KeywordTok}[1]{\textcolor[rgb]{0.13,0.29,0.53}{\textbf{#1}}}
\newcommand{\NormalTok}[1]{#1}
\newcommand{\OperatorTok}[1]{\textcolor[rgb]{0.81,0.36,0.00}{\textbf{#1}}}
\newcommand{\OtherTok}[1]{\textcolor[rgb]{0.56,0.35,0.01}{#1}}
\newcommand{\PreprocessorTok}[1]{\textcolor[rgb]{0.56,0.35,0.01}{\textit{#1}}}
\newcommand{\RegionMarkerTok}[1]{#1}
\newcommand{\SpecialCharTok}[1]{\textcolor[rgb]{0.00,0.00,0.00}{#1}}
\newcommand{\SpecialStringTok}[1]{\textcolor[rgb]{0.31,0.60,0.02}{#1}}
\newcommand{\StringTok}[1]{\textcolor[rgb]{0.31,0.60,0.02}{#1}}
\newcommand{\VariableTok}[1]{\textcolor[rgb]{0.00,0.00,0.00}{#1}}
\newcommand{\VerbatimStringTok}[1]{\textcolor[rgb]{0.31,0.60,0.02}{#1}}
\newcommand{\WarningTok}[1]{\textcolor[rgb]{0.56,0.35,0.01}{\textbf{\textit{#1}}}}
\usepackage{graphicx,grffile}
\makeatletter
\def\maxwidth{\ifdim\Gin@nat@width>\linewidth\linewidth\else\Gin@nat@width\fi}
\def\maxheight{\ifdim\Gin@nat@height>\textheight\textheight\else\Gin@nat@height\fi}
\makeatother
% Scale images if necessary, so that they will not overflow the page
% margins by default, and it is still possible to overwrite the defaults
% using explicit options in \includegraphics[width, height, ...]{}
\setkeys{Gin}{width=\maxwidth,height=\maxheight,keepaspectratio}
% Set default figure placement to htbp
\makeatletter
\def\fps@figure{htbp}
\makeatother
\setlength{\emergencystretch}{3em} % prevent overfull lines
\providecommand{\tightlist}{%
  \setlength{\itemsep}{0pt}\setlength{\parskip}{0pt}}
\setcounter{secnumdepth}{-\maxdimen} % remove section numbering

\title{Party Perception's Role in Elections}
\author{}
\date{\vspace{-2.5em}}

\begin{document}
\maketitle

This is an \href{http://rmarkdown.rstudio.com}{R Markdown} Notebook.

\#Perception in elections

In 1980 it was competence, leadership in 1984, patriotism in 1988,
trustworthiness in 1992 and 1996, integrity and leadership ability in
2000, decisiveness and leadership in 2004, and Knowledge and temperament
in 2008---these candidate traits have been deciding factors in elections
According to the Inter-university Consortium for Political and Social
Research, candidate traits can make or break an election outcome.

While the issues are important to many voters, the images of candidates
that voters shape in their minds are heavily influenced by voters'
perception of a candidate's personal qualities. You're more likely to
vote for a candidate who you think is honest---or what some might call
moral---than a candidate you see as untrustworthy. Or, If you're on the
fence you're likely to select the candidate that you think is a better
leader. A voter's vote for president is a very personal one, and traits
play a role in shaping that vote.

The American National Election Studies (ANES) are surveys of voters in
the U.S. on a national scale. For each presidential election since 1980,
ANES has collected information about the personality traits of
presidential candidates, by asking how well 8 personality traits
describe republican and democratic candidates. The traits surveyed where
Intelligence, Compassion, Decency, Inspiring, Knowledge, Morals,
Leadership, and Caring, and respondents could rank each trait on a scale
of one to four, one being ``Extremely well'' and four being ``Not well
at all.''

With this data, we explore not only the perceptions of individuals in
individual races, but general perceptions b

\#\#Set Up

Install Packages and load the raw data

Create two output data files, one for the perceptions of the democratic
candidate, another for the perceptions of the republican candidate. Both
these datasets will also include the vote decision, to see how connected
perceptions are to voter turnouts

\begin{Shaded}
\begin{Highlighting}[]
\NormalTok{Election_years=}\KeywordTok{as.character}\NormalTok{(}\KeywordTok{seq}\NormalTok{(}\DecValTok{1952}\NormalTok{, }\DecValTok{2016}\NormalTok{, }\DecValTok{4}\NormalTok{))}

\NormalTok{anes_dem=anes_dat}\OperatorTok
\StringTok{  }\KeywordTok{mutate}\NormalTok{(}
    \DataTypeTok{year=}\KeywordTok{as_factor}\NormalTok{(VCF0004),}
    \DataTypeTok{vote=}\KeywordTok{as_factor}\NormalTok{(VCF0706),}
    \DataTypeTok{Intelligent =}   \KeywordTok{as_factor}\NormalTok{(  VCF0350 ),}
    \DataTypeTok{Compassionate   =}   \KeywordTok{as_factor}\NormalTok{(  VCF0351 ),}
    \DataTypeTok{Decent  =}   \KeywordTok{as_factor}\NormalTok{(  VCF0352 ),}
    \DataTypeTok{Inspiring   =}   \KeywordTok{as_factor}\NormalTok{(  VCF0353 ),}
    \DataTypeTok{Knowledgeable   =}   \KeywordTok{as_factor}\NormalTok{(  VCF0354 ),}
    \DataTypeTok{Moral   =}   \KeywordTok{as_factor}\NormalTok{(  VCF0355 ),}
    \DataTypeTok{Leadership  =}   \KeywordTok{as_factor}\NormalTok{(  VCF0356 ),}
    \DataTypeTok{Cares   =}   \KeywordTok{as_factor}\NormalTok{(  VCF0357 )}
\NormalTok{  )}\OperatorTok
\StringTok{  }\KeywordTok{filter}\NormalTok{(year }\OperatorTok\StringTok{ }\NormalTok{Election_years)}

\NormalTok{anes_dem =}\StringTok{ }\NormalTok{anes_dem}\OperatorTok
\StringTok{  }\KeywordTok{select}\NormalTok{(year, vote, Intelligent,   Compassionate,  Decent, Inspiring,  Knowledgeable,  Moral,  Leadership, Cares)}\OperatorTok
\StringTok{  }\KeywordTok{gather}\NormalTok{(}\DataTypeTok{key =} \StringTok{"quality"}\NormalTok{, }\DataTypeTok{value =} \StringTok{"measure"}\NormalTok{, }\OperatorTok{-}\NormalTok{year, }\OperatorTok{-}\NormalTok{vote, }\DataTypeTok{na.rm =} \OtherTok{TRUE}\NormalTok{,}
  \DataTypeTok{convert =} \OtherTok{FALSE}\NormalTok{, }\DataTypeTok{factor_key =} \OtherTok{FALSE}\NormalTok{)}
\end{Highlighting}
\end{Shaded}

\begin{verbatim}
## Warning: attributes are not identical across measure variables;
## they will be dropped
\end{verbatim}

\begin{Shaded}
\begin{Highlighting}[]
\KeywordTok{save}\NormalTok{(anes_dem, }\DataTypeTok{file=}\StringTok{"../output/data_dem.RData"}\NormalTok{)}


\NormalTok{anes_rep=anes_dat}\OperatorTok
\StringTok{  }\KeywordTok{mutate}\NormalTok{(}
    \DataTypeTok{year=}\KeywordTok{as_factor}\NormalTok{(VCF0004),}
    \DataTypeTok{vote=}\KeywordTok{as_factor}\NormalTok{(VCF0706),}
    \DataTypeTok{Intelligent =}   \KeywordTok{as_factor}\NormalTok{(  VCF0362 ),}
    \DataTypeTok{Compassionate   =}   \KeywordTok{as_factor}\NormalTok{(  VCF0363 ),}
    \DataTypeTok{Decent  =}   \KeywordTok{as_factor}\NormalTok{(  VCF0364 ),}
    \DataTypeTok{Inspiring   =}   \KeywordTok{as_factor}\NormalTok{(  VCF0365 ),}
    \DataTypeTok{Knowledgeable   =}   \KeywordTok{as_factor}\NormalTok{(  VCF0366 ),}
    \DataTypeTok{Moral   =}   \KeywordTok{as_factor}\NormalTok{(  VCF0367 ),}
    \DataTypeTok{Leadership  =}   \KeywordTok{as_factor}\NormalTok{(  VCF0368 ),}
    \DataTypeTok{Cares   =}   \KeywordTok{as_factor}\NormalTok{(  VCF0369 )}
\NormalTok{  )}\OperatorTok
\StringTok{  }\KeywordTok{filter}\NormalTok{(year }\OperatorTok\StringTok{ }\NormalTok{Election_years)}


\NormalTok{anes_rep =}\StringTok{ }\NormalTok{anes_rep }\OperatorTok\StringTok{ }
\StringTok{  }\KeywordTok{select}\NormalTok{(year, vote, Intelligent,   Compassionate,  Decent, Inspiring,  Knowledgeable,  Moral,  Leadership, Cares) }\OperatorTok
\StringTok{  }\KeywordTok{gather}\NormalTok{(}\DataTypeTok{key =} \StringTok{"quality"}\NormalTok{, }\DataTypeTok{value =} \StringTok{"measure"}\NormalTok{, }\OperatorTok{-}\NormalTok{year, }\OperatorTok{-}\NormalTok{vote, }\DataTypeTok{na.rm =} \OtherTok{TRUE}\NormalTok{,}
  \DataTypeTok{convert =} \OtherTok{FALSE}\NormalTok{, }\DataTypeTok{factor_key =} \OtherTok{FALSE}\NormalTok{)}
\end{Highlighting}
\end{Shaded}

\begin{verbatim}
## Warning: attributes are not identical across measure variables;
## they will be dropped
\end{verbatim}

\begin{Shaded}
\begin{Highlighting}[]
\KeywordTok{save}\NormalTok{(anes_rep, }\DataTypeTok{file=}\StringTok{"../output/data_rep.RData"}\NormalTok{)}

\NormalTok{anes_all <-}\StringTok{ }\NormalTok{gdata}\OperatorTok{::}\KeywordTok{combine}\NormalTok{(anes_dem, anes_rep)}

\KeywordTok{save}\NormalTok{(anes_all, }\DataTypeTok{file=}\StringTok{"../output/data_all.RData"}\NormalTok{)}
\end{Highlighting}
\end{Shaded}

\#\#What qualities describe do voters use to describe Democrats and
Republicans? Outside of the candidate, there is first the question of
the party as a whole. Turn on Fox News, and you might hear Tucker
Carlson describe democrats as ``vindictive'' --- meaning immoral and
lacking decency--- or ``snowflakes'' --- meaning weak or overly
sensitive. Switch to MSNBC and you'll hear about republican ``hacks''---
lacking leadership and compassion--- and ``bigots'' --- missing
knowledge and honesty.

But there is a better way to learn about what qualifies to define a
party.

With ANES data we can look at the opinions of candidates' traits from
either party. On the graphs below, a series of grouped bar charts show
the percentage of respondents who have picked each ranking, split by the
8 different characteristics. Within any given characteristic there is
approximately the same number of respondents for the Democratic
candidate and the Republican candidate, however, some characteristics
have more respondents overall than others. For that reason, each bar is
labeled with the raw number of respondents who compose that percentage.

\begin{Shaded}
\begin{Highlighting}[]
\CommentTok{#aes(label=scales::percent(round(..prop..,2)), y=..prop..)}

\NormalTok{p<-}\StringTok{ }\KeywordTok{ggplot}\NormalTok{(anes_all, }\DataTypeTok{mapping =} \KeywordTok{aes}\NormalTok{(}\DataTypeTok{fill=}\NormalTok{source, }\DataTypeTok{x=}\NormalTok{measure, }\DataTypeTok{group=}\KeywordTok{interaction}\NormalTok{(source, quality), }\DataTypeTok{y=}\NormalTok{ (..prop..))) }\OperatorTok{+}\StringTok{ }
\StringTok{  }\KeywordTok{geom_bar}\NormalTok{(}\DataTypeTok{position=}\StringTok{"dodge"}\NormalTok{) }\OperatorTok{+}
\StringTok{  }\KeywordTok{scale_fill_manual}\NormalTok{(}\DataTypeTok{values =} \KeywordTok{c}\NormalTok{(}\StringTok{"#054C70"}\NormalTok{,}\StringTok{"#b5040a"}\NormalTok{)) }\OperatorTok{+}
\StringTok{  }\KeywordTok{geom_text}\NormalTok{(}\KeywordTok{aes}\NormalTok{(}\DataTypeTok{label=}\NormalTok{..count.., }
                \DataTypeTok{y=}\NormalTok{..prop..), }\DataTypeTok{stat=}\StringTok{"count"}\NormalTok{, }\DataTypeTok{vjust=}\OperatorTok{-}\NormalTok{.}\DecValTok{5}\NormalTok{, }\DataTypeTok{position=}\KeywordTok{position_dodge}\NormalTok{(.}\DecValTok{9}\NormalTok{)) }\OperatorTok{+}
\StringTok{  }\KeywordTok{scale_y_continuous}\NormalTok{(}\DataTypeTok{limits=}\KeywordTok{c}\NormalTok{(}\DecValTok{0}\NormalTok{,}\DecValTok{1}\NormalTok{),}\DataTypeTok{labels =}\NormalTok{ scales}\OperatorTok{::}\NormalTok{percent) }
  
\NormalTok{p }\OperatorTok{+}\StringTok{ }\KeywordTok{facet_wrap}\NormalTok{(}\OperatorTok{~}\NormalTok{quality, }\DataTypeTok{ncol=}\DecValTok{2}\NormalTok{)}
\end{Highlighting}
\end{Shaded}

\includegraphics{prelim_exploration_files/figure-latex/unnamed-chunk-2-1.pdf}
There are some interesting things to note here. A larger percentage of
respondents identify democrats, rather than republican candidates, as
extremely or quite caring (63 percent of Democrats candidates are ranked
this while only 44 percent of Republicans are), compassionate (76 versus
60 percent), and Intelligent (87 versus 77 percent).

Whereas republicans beat out democrats, in leadership (61 versus 55
percent) and morals (76 versus 68). You'll notice that these margins are
much more narrow than those that are democratic lead.

A few qualities were too close to distinguish (less than or equal to 5
percent gaps), that being Decent (89 percent of Democrats, 85 percent of
Republicans), knowledgeable (77, 72), and inspiring (50, 45).

This visualization allows a less partisan way than cable news to think
about how characteristics define different parties. And some of the
takeaways make a lot of sense considering what is known about the party
breakdown of the United States. Take the trait ``inspiring'' for
example, inspiration ability is incredibly objective, but it is one of
the traits candidates try to present themselves as most often in
elections. It isn't surprising that that is split in half and half, as
it is only created by opinions and charisma rather than actions a
candidate takes (online leadership or intelligence for example).

While we do see some differences in overall party perception, the
largest gap in opinion we see is 19 percentage points. This is
significant considering the thousands of responses that we have
available, but isn't massive, suggesting that overall differences in
party perception may be caused more by personal bias than actual public
opinion.

\#\#How has voters' perceptions around a party's candidates changed over
time?

This first example gives us an understanding of the sum of all years but
political parties can often change dramatically over time, and fluctuate
based on a particular race. To break this down further let's look at how
the perception of some of these characteristics has changed over 28
years of ANES data available.

During that 28 year timespan, complete data is only available for three
out of our eight characteristics, Knowledgeable, Leadership, and Morals.
With this, we can make a chart similar to the first one.

\begin{Shaded}
\begin{Highlighting}[]
\NormalTok{time_plt <-}\StringTok{ }\KeywordTok{ggplot}\NormalTok{(}\KeywordTok{subset}\NormalTok{(anes_all, quality }\OperatorTok\StringTok{ }\KeywordTok{c}\NormalTok{(}\StringTok{"Knowledgeable"}\NormalTok{, }\StringTok{"Leadership"}\NormalTok{, }\StringTok{"Moral"}\NormalTok{)), }\DataTypeTok{mapping =} \KeywordTok{aes}\NormalTok{(}\DataTypeTok{fill=}\NormalTok{source, }\DataTypeTok{x=}\NormalTok{measure, }\DataTypeTok{group=}\KeywordTok{interaction}\NormalTok{(source, quality), }\DataTypeTok{y=}\NormalTok{ (..prop..))) }\OperatorTok{+}\StringTok{ }
\StringTok{  }\KeywordTok{geom_bar}\NormalTok{(}\DataTypeTok{position=}\StringTok{"dodge"}\NormalTok{) }\OperatorTok{+}
\StringTok{  }\KeywordTok{scale_fill_manual}\NormalTok{(}\DataTypeTok{values =} \KeywordTok{c}\NormalTok{(}\StringTok{"#054C70"}\NormalTok{,}\StringTok{"#b5040a"}\NormalTok{)) }\OperatorTok{+}
\StringTok{  }\KeywordTok{geom_text}\NormalTok{(}\KeywordTok{aes}\NormalTok{(}\DataTypeTok{label=}\NormalTok{..count.., }
                \DataTypeTok{y=}\NormalTok{..prop..), }\DataTypeTok{stat=}\StringTok{"count"}\NormalTok{, }\DataTypeTok{angle =} \DecValTok{90}\NormalTok{, }\DataTypeTok{hjust=}\OperatorTok{-}\NormalTok{.}\DecValTok{25}\NormalTok{, }\DataTypeTok{position=}\KeywordTok{position_dodge}\NormalTok{(.}\DecValTok{9}\NormalTok{)) }\OperatorTok{+}
\StringTok{  }\KeywordTok{scale_y_continuous}\NormalTok{(}\DataTypeTok{limits=}\KeywordTok{c}\NormalTok{(}\DecValTok{0}\NormalTok{,}\DecValTok{1}\NormalTok{),}\DataTypeTok{labels =}\NormalTok{ scales}\OperatorTok{::}\NormalTok{percent) }\OperatorTok{+}
\StringTok{  }\KeywordTok{theme}\NormalTok{(}\DataTypeTok{axis.text.x =} \KeywordTok{element_text}\NormalTok{(}\DataTypeTok{angle =} \DecValTok{90}\NormalTok{, }\DataTypeTok{vjust =} \FloatTok{0.5}\NormalTok{, }\DataTypeTok{hjust=}\DecValTok{1}\NormalTok{))}


\NormalTok{time_plt }\OperatorTok{+}\StringTok{ }\KeywordTok{facet_grid}\NormalTok{(quality }\OperatorTok{~}\StringTok{ }\NormalTok{year)}
\end{Highlighting}
\end{Shaded}

\includegraphics{prelim_exploration_files/figure-latex/unnamed-chunk-3-1.pdf}

\begin{Shaded}
\begin{Highlighting}[]
\NormalTok{time <-}\StringTok{ }\NormalTok{p }\OperatorTok{+}\StringTok{ }\KeywordTok{facet_grid}\NormalTok{(quality }\OperatorTok{~}\StringTok{ }\NormalTok{year)}
\end{Highlighting}
\end{Shaded}

\begin{Shaded}
\begin{Highlighting}[]
\NormalTok{time_data <-}\StringTok{ }\KeywordTok{layer_data}\NormalTok{(time, }\DecValTok{1}\NormalTok{)}

\CommentTok{#filter out years without data}
\NormalTok{time_data <-}\StringTok{ }\KeywordTok{filter}\NormalTok{(time_data, (xmax }\OperatorTok{<}\StringTok{ }\FloatTok{2.5} \OperatorTok{&}\StringTok{ }\KeywordTok{as.integer}\NormalTok{(PANEL) }\OperatorTok{>}\StringTok{ }\DecValTok{40}\NormalTok{))}
\NormalTok{time_data}\OperatorTok{$}\NormalTok{year <-}\StringTok{ }\KeywordTok{as.integer}\NormalTok{(time_data}\OperatorTok{$}\NormalTok{PANEL)}
\NormalTok{time_data}\OperatorTok{$}\NormalTok{quality <-}\StringTok{ }\KeywordTok{as.integer}\NormalTok{(time_data}\OperatorTok{$}\NormalTok{PANEL)}
\NormalTok{time_data}\OperatorTok{$}\NormalTok{panel_num <-}\StringTok{ }\KeywordTok{as.integer}\NormalTok{(time_data}\OperatorTok{$}\NormalTok{PANEL)}

\CommentTok{#add year}

\ControlFlowTok{for}\NormalTok{ (x }\ControlFlowTok{in} \KeywordTok{seq}\NormalTok{(}\DecValTok{0}\NormalTok{,}\DecValTok{7}\NormalTok{))\{}
\NormalTok{  time_data}\OperatorTok{$}\NormalTok{year[time_data}\OperatorTok{$}\NormalTok{panel_num }\OperatorTok\StringTok{ }\KeywordTok{seq}\NormalTok{(}\DecValTok{41}\OperatorTok{+}\NormalTok{x, }\DecValTok{64}\NormalTok{, }\DecValTok{8}\NormalTok{)] <-}\StringTok{ }\NormalTok{(}\DecValTok{1980} \OperatorTok{+}\StringTok{ }\DecValTok{4}\OperatorTok{*}\NormalTok{x)}
                      
  \CommentTok{#time_data$year[as.integer(time_data$PANEL)==seq(41+x, 64, 8)] <- (1980 + 4*x)}
\NormalTok{\}}

\CommentTok{#add quality}

\NormalTok{time_data}\OperatorTok{$}\NormalTok{quality[time_data}\OperatorTok{$}\NormalTok{panel_num }\OperatorTok\StringTok{ }\KeywordTok{seq}\NormalTok{(}\DecValTok{41}\NormalTok{, }\DecValTok{48}\NormalTok{)] <-}\StringTok{ "Knowledgeable"}
\NormalTok{time_data}\OperatorTok{$}\NormalTok{quality[time_data}\OperatorTok{$}\NormalTok{panel_num }\OperatorTok\StringTok{ }\KeywordTok{seq}\NormalTok{(}\DecValTok{49}\NormalTok{, }\DecValTok{56}\NormalTok{ )] <-}\StringTok{ "Leadership"}
\NormalTok{time_data}\OperatorTok{$}\NormalTok{quality[time_data}\OperatorTok{$}\NormalTok{panel_num }\OperatorTok\StringTok{ }\KeywordTok{seq}\NormalTok{(}\DecValTok{57}\NormalTok{, }\DecValTok{64}\NormalTok{ )] <-}\StringTok{ "Moral"}

\NormalTok{time_data <-}\StringTok{ }\NormalTok{time_data }\OperatorTok\StringTok{ }\KeywordTok{group_by}\NormalTok{(quality, fill, year) }\OperatorTok\StringTok{  }\KeywordTok{summarize}\NormalTok{(}\DataTypeTok{percent =} \KeywordTok{sum}\NormalTok{(y))}
\end{Highlighting}
\end{Shaded}

\begin{verbatim}
## `summarise()` has grouped output by 'quality', 'fill'. You can override using the `.groups` argument.
\end{verbatim}

This graph allows us to explore every ranking across all the years, but
it's hard to see trends. Let's look Instead at a line chart plotting the
proportion of respondents that think a candidate exemplifies each
quality extremely or quite well.

\begin{Shaded}
\begin{Highlighting}[]
 \KeywordTok{ggplot}\NormalTok{(time_data, }\KeywordTok{aes}\NormalTok{(}\DataTypeTok{x=}\NormalTok{year, }\DataTypeTok{y=}\NormalTok{percent, }\DataTypeTok{group=}\NormalTok{fill, }\DataTypeTok{color=}\NormalTok{fill)) }\OperatorTok{+}
\StringTok{  }\KeywordTok{geom_line}\NormalTok{() }\OperatorTok{+}
\StringTok{  }\KeywordTok{facet_wrap}\NormalTok{(}\OperatorTok{~}\NormalTok{quality, }\DataTypeTok{ncol=}\DecValTok{1}\NormalTok{) }\OperatorTok{+}
\StringTok{  }\KeywordTok{scale_y_continuous}\NormalTok{(}\DataTypeTok{limits=}\KeywordTok{c}\NormalTok{(}\DecValTok{0}\NormalTok{,}\DecValTok{1}\NormalTok{),}\DataTypeTok{labels =}\NormalTok{ scales}\OperatorTok{::}\NormalTok{percent) }\OperatorTok{+}
\StringTok{  }\KeywordTok{scale_color_manual}\NormalTok{(}\DataTypeTok{values =} \KeywordTok{c}\NormalTok{(}\StringTok{"#054C70"}\NormalTok{,}\StringTok{"#b5040a"}\NormalTok{)) }\OperatorTok{+}\StringTok{ }
\StringTok{ }\KeywordTok{scale_x_continuous}\NormalTok{(}\DataTypeTok{breaks =} \KeywordTok{seq}\NormalTok{(}\DecValTok{1980}\NormalTok{, }\DecValTok{2008}\NormalTok{, }\DecValTok{4}\NormalTok{))}
\end{Highlighting}
\end{Shaded}

\includegraphics{prelim_exploration_files/figure-latex/unnamed-chunk-5-1.pdf}
Some characteristics we see are rather stable over time, like the
knowledge that you've already established is also pretty Samoan between
Democratic and Republican candidates.

Perhaps the most interesting case is that of morals; a large hit of
morals happens in the 1992 and 1966 elections. This aligns with a report
for the ICPSR which named trustworthiness as the deciding factor in both
those elections, a factor often associated closely with morality.

\#\#What is the relationship between character perception and votes?

Last but not least, let's Introduce how these perceptions impact voting
outcomes. Lets start with a highlevel look at the Data.

\begin{Shaded}
\begin{Highlighting}[]
\NormalTok{votes <-}\StringTok{ }\KeywordTok{ggplot}\NormalTok{(}\KeywordTok{subset}\NormalTok{(anes_all, vote }\OperatorTok\StringTok{ }\KeywordTok{c}\NormalTok{(}\StringTok{"1. Democrat"}\NormalTok{, }\StringTok{"2. Republican"}\NormalTok{)), }
                \DataTypeTok{mapping =} \KeywordTok{aes}\NormalTok{(}\DataTypeTok{fill=}\NormalTok{source, }\DataTypeTok{x=}\NormalTok{measure, }\DataTypeTok{group=}\KeywordTok{interaction}\NormalTok{(source, quality), }\DataTypeTok{y=}\NormalTok{ (..prop..))) }\OperatorTok{+}\StringTok{ }
\StringTok{  }\KeywordTok{geom_bar}\NormalTok{(}\DataTypeTok{position=}\StringTok{"dodge"}\NormalTok{) }\OperatorTok{+}
\StringTok{  }\KeywordTok{scale_fill_manual}\NormalTok{(}\DataTypeTok{values =} \KeywordTok{c}\NormalTok{(}\StringTok{"#054C70"}\NormalTok{,}\StringTok{"#b5040a"}\NormalTok{)) }\OperatorTok{+}
\StringTok{  }\KeywordTok{geom_text}\NormalTok{(}\KeywordTok{aes}\NormalTok{(}\DataTypeTok{label=}\NormalTok{..count.., }
                \DataTypeTok{y=}\NormalTok{..prop..), }\DataTypeTok{stat=}\StringTok{"count"}\NormalTok{, }\DataTypeTok{vjust=}\OperatorTok{-}\NormalTok{.}\DecValTok{5}\NormalTok{, }\DataTypeTok{position=}\KeywordTok{position_dodge}\NormalTok{(.}\DecValTok{9}\NormalTok{)) }\OperatorTok{+}
\StringTok{  }\KeywordTok{scale_y_continuous}\NormalTok{(}\DataTypeTok{limits=}\KeywordTok{c}\NormalTok{(}\DecValTok{0}\NormalTok{,}\DecValTok{1}\NormalTok{),}\DataTypeTok{labels =}\NormalTok{ scales}\OperatorTok{::}\NormalTok{percent)  }\OperatorTok{+}\StringTok{ }
\StringTok{  }\KeywordTok{facet_grid}\NormalTok{(quality}\OperatorTok{~}\StringTok{ }\NormalTok{vote) }
  
  

\NormalTok{votes}
\end{Highlighting}
\end{Shaded}

\includegraphics{prelim_exploration_files/figure-latex/unnamed-chunk-6-1.pdf}

\begin{Shaded}
\begin{Highlighting}[]
\NormalTok{get_conditonal <-}\StringTok{ }\ControlFlowTok{function}\NormalTok{(qual, }\DataTypeTok{x=}\DecValTok{2}\NormalTok{) \{}
  
\NormalTok{  conditions <-}\StringTok{ }\NormalTok{anes_all }\OperatorTok\StringTok{ }\KeywordTok{filter}\NormalTok{(vote }\OperatorTok\StringTok{ }\KeywordTok{c}\NormalTok{(}\StringTok{"1. Democrat"}\NormalTok{, }\StringTok{"2. Republican"}\NormalTok{), measure }\OperatorTok\StringTok{ }\KeywordTok{c}\NormalTok{(}\StringTok{"1. Extremely well"}\NormalTok{,}\StringTok{"2. Quite well"}\NormalTok{ ), quality}\OperatorTok{==}\NormalTok{qual)}

\NormalTok{  test <-}\StringTok{ }\KeywordTok{table}\NormalTok{(conditions}\OperatorTok{$}\NormalTok{vote, conditions}\OperatorTok{$}\NormalTok{source)[}\OperatorTok{-}\KeywordTok{c}\NormalTok{(}\DecValTok{1}\NormalTok{,}\DecValTok{4}\NormalTok{,}\DecValTok{5}\NormalTok{,}\DecValTok{6}\NormalTok{),]}

  \CommentTok{# formattable(as.data.frame.matrix(test))}
  
  \ControlFlowTok{if}\NormalTok{ (x }\OperatorTok{>}\StringTok{ }\DecValTok{0} \OperatorTok{&}\StringTok{ }\NormalTok{x}\OperatorTok{<}\DecValTok{3}\NormalTok{) \{}
    \KeywordTok{formattable}\NormalTok{(}\KeywordTok{as.data.frame.matrix}\NormalTok{(}\KeywordTok{round}\NormalTok{(}\KeywordTok{prop.table}\NormalTok{(test,x), }\DecValTok{3}\NormalTok{)), }
              \KeywordTok{list}\NormalTok{(}\DataTypeTok{anes_dem =} \KeywordTok{color_bar}\NormalTok{(}\StringTok{"#adc0c9"}\NormalTok{),}
                   \DataTypeTok{anes_rep =} \KeywordTok{color_bar}\NormalTok{(}\StringTok{"#ebb2b4"}\NormalTok{)))}
\NormalTok{  \} }\ControlFlowTok{else}\NormalTok{ \{}
    \KeywordTok{formattable}\NormalTok{(}\KeywordTok{as.data.frame.matrix}\NormalTok{(}\KeywordTok{round}\NormalTok{(}\KeywordTok{prop.table}\NormalTok{(test), }\DecValTok{3}\NormalTok{)), }
              \KeywordTok{list}\NormalTok{(}\DataTypeTok{anes_dem =} \KeywordTok{color_bar}\NormalTok{(}\StringTok{"#adc0c9"}\NormalTok{),}
                   \DataTypeTok{anes_rep =} \KeywordTok{color_bar}\NormalTok{(}\StringTok{"#ebb2b4"}\NormalTok{)))}
\NormalTok{  \}}
  
\NormalTok{\}}
\end{Highlighting}
\end{Shaded}

Some trends are obvious here, but with some conditional prob we can see
deeper in to relationships.

Take the characteristic caring for an example. This table shows the
conditional probabilities, that given that you think a candidate cares,
quite Well or extremely well, you will vote republican or democrat.

\begin{Shaded}
\begin{Highlighting}[]
\KeywordTok{get_conditonal}\NormalTok{(}\StringTok{"Cares"}\NormalTok{ )}
\end{Highlighting}
\end{Shaded}

anes\_dem

anes\_rep

\begin{enumerate}
\def\labelenumi{\arabic{enumi}.}
\tightlist
\item
  Democrat

  {0.734}

  {0.184}

  \begin{enumerate}
  \def\labelenumii{\arabic{enumii}.}
  \setcounter{enumii}{1}
  \tightlist
  \item
    Republican

    {0.266}

    {0.816}
  \end{enumerate}
\end{enumerate}

These tables can be interpreted to mean that if you think the democratic
candidate is caring, there is a 73\% percent chance you will vote
democratic. A similar trend is seen for republican candidates, where any
given person has an 81 percent chance of voting republican if you think
the Republican candidate is caring.

This gets more interesting when you begin to compare, who a respondent
does not vote for. For instance, the probability that given you think
the Republican candidate is caring but you still decide to vote for the
Democratic candidate is dramatically lower at 18 and 27\% vice versa.
This suggests that when assessing candidate voting, voters tend to
gravitate towards the extremes, in other words, there are few examples
when both candidates are perceived as embodying a characteristic.

That pattern follows for all traits.

\begin{Shaded}
\begin{Highlighting}[]
\KeywordTok{get_conditonal}\NormalTok{(}\StringTok{"Moral"}\NormalTok{)}
\end{Highlighting}
\end{Shaded}

anes\_dem

anes\_rep

\begin{enumerate}
\def\labelenumi{\arabic{enumi}.}
\tightlist
\item
  Democrat

  {0.634}

  {0.404}

  \begin{enumerate}
  \def\labelenumii{\arabic{enumii}.}
  \setcounter{enumii}{1}
  \tightlist
  \item
    Republican

    {0.366}

    {0.596}
  \end{enumerate}
\end{enumerate}

\begin{Shaded}
\begin{Highlighting}[]
\KeywordTok{get_conditonal}\NormalTok{(}\StringTok{"Compassionate"}\NormalTok{)}
\end{Highlighting}
\end{Shaded}

anes\_dem

anes\_rep

\begin{enumerate}
\def\labelenumi{\arabic{enumi}.}
\tightlist
\item
  Democrat

  {0.625}

  {0.278}

  \begin{enumerate}
  \def\labelenumii{\arabic{enumii}.}
  \setcounter{enumii}{1}
  \tightlist
  \item
    Republican

    {0.375}

    {0.722}
  \end{enumerate}
\end{enumerate}

\begin{Shaded}
\begin{Highlighting}[]
\KeywordTok{get_conditonal}\NormalTok{(}\StringTok{"Intelligent"}\NormalTok{)}
\end{Highlighting}
\end{Shaded}

anes\_dem

anes\_rep

\begin{enumerate}
\def\labelenumi{\arabic{enumi}.}
\tightlist
\item
  Democrat

  {0.572}

  {0.408}

  \begin{enumerate}
  \def\labelenumii{\arabic{enumii}.}
  \setcounter{enumii}{1}
  \tightlist
  \item
    Republican

    {0.428}

    {0.592}
  \end{enumerate}
\end{enumerate}

\begin{Shaded}
\begin{Highlighting}[]
\KeywordTok{get_conditonal}\NormalTok{(}\StringTok{"Decent"}\NormalTok{)}
\end{Highlighting}
\end{Shaded}

anes\_dem

anes\_rep

\begin{enumerate}
\def\labelenumi{\arabic{enumi}.}
\tightlist
\item
  Democrat

  {0.483}

  {0.365}

  \begin{enumerate}
  \def\labelenumii{\arabic{enumii}.}
  \setcounter{enumii}{1}
  \tightlist
  \item
    Republican

    {0.517}

    {0.635}
  \end{enumerate}
\end{enumerate}

\begin{Shaded}
\begin{Highlighting}[]
\KeywordTok{get_conditonal}\NormalTok{(}\StringTok{"Inspiring"}\NormalTok{)}
\end{Highlighting}
\end{Shaded}

anes\_dem

anes\_rep

\begin{enumerate}
\def\labelenumi{\arabic{enumi}.}
\tightlist
\item
  Democrat

  {0.738}

  {0.246}

  \begin{enumerate}
  \def\labelenumii{\arabic{enumii}.}
  \setcounter{enumii}{1}
  \tightlist
  \item
    Republican

    {0.262}

    {0.754}
  \end{enumerate}
\end{enumerate}

\begin{Shaded}
\begin{Highlighting}[]
\KeywordTok{get_conditonal}\NormalTok{(}\StringTok{"Knowledgeable"}\NormalTok{)}
\end{Highlighting}
\end{Shaded}

anes\_dem

anes\_rep

\begin{enumerate}
\def\labelenumi{\arabic{enumi}.}
\tightlist
\item
  Democrat

  {0.596}

  {0.413}

  \begin{enumerate}
  \def\labelenumii{\arabic{enumii}.}
  \setcounter{enumii}{1}
  \tightlist
  \item
    Republican

    {0.404}

    {0.587}
  \end{enumerate}
\end{enumerate}

\begin{Shaded}
\begin{Highlighting}[]
\KeywordTok{get_conditonal}\NormalTok{(}\StringTok{"Leadership"}\NormalTok{)}
\end{Highlighting}
\end{Shaded}

anes\_dem

anes\_rep

\begin{enumerate}
\def\labelenumi{\arabic{enumi}.}
\tightlist
\item
  Democrat

  {0.788}

  {0.31}

  \begin{enumerate}
  \def\labelenumii{\arabic{enumii}.}
  \setcounter{enumii}{1}
  \tightlist
  \item
    Republican

    {0.212}

    {0.69}
  \end{enumerate}
\end{enumerate}

A final takeaway from this analysis is just how important candidate
characteristics are to voting outcomes. The high and low conditional
probabilities suggest that they are well connected to final voter
outcomes, and can be an explanation for much of the pandering and
character argument we see prevalent in elections.

\end{document}
